
%% Verze pro jednostranný tisk:
% Okraje: levý 40mm, pravý 25mm, horní a dolní 25mm
% (ale pozor, LaTeX si sám přidává 1in)
\documentclass[12pt,a4paper]{report}

% \openright zařídí, aby následující text začínal na pravé straně knihy
\let\openright=\clearpage

%% Pokud tiskneme oboustranně:
% \documentclass[12pt,a4paper,twoside,openright]{report}
% \setlength\textwidth{145mm}
% \setlength\textheight{247mm}
% \setlength\oddsidemargin{14.2mm}
% \setlength\evensidemargin{0mm}
% \setlength\topmargin{0mm}
% \setlength\headsep{0mm}
% \setlength\headheight{0mm}
% \let\openright=\cleardoublepage

%% Vytváříme PDF/A-2u
\usepackage[a-2u]{pdfx}

%% Přepneme na českou sazbu a fonty Latin Modern
\usepackage[slovak]{babel}
\usepackage{lmodern}
\usepackage[IL2]{fontenc}%T1
\usepackage{textcomp}
\usepackage{hyperref}

\usepackage{float}
\usepackage{subfigure}

%% Použité kódování znaků: obvykle latin2, cp1250 nebo utf8:
\usepackage[utf8]{inputenc}

%%% Další užitečné balíčky (jsou součástí běžných distribucí LaTeXu)
\usepackage{amsmath}        % rozšíření pro sazbu matematiky
\usepackage{amsfonts}       % matematické fonty
\usepackage{amsthm}         % sazba vět, definic apod.

%bolo treba vypnut kvoli rozdelovaniu
%\usepackage{bbding}         % balíček s nejrůznějšími symboly
% (čtverečky, hvězdičky, tužtičky, nůžtičky, ...)
\usepackage{bm}             % tučné symboly (příkaz \bm)
\usepackage{graphicx}       % vkládání obrázků
\usepackage{fancyvrb}       % vylepšené prostředí pro strojové písmo
\usepackage{indentfirst}    % zavede odsazení 1. odstavce kapitoly
%\usepackage{natbib}         % zajištuje možnost odkazovat na literaturu
% stylem AUTOR (ROK), resp. AUTOR [ČÍSLO]
\usepackage[nottoc]{tocbibind} % zajistí přidání seznamu literatury,
% obrázků a tabulek do obsahu
\usepackage{icomma}         % inteligetní čárka v matematickém módu
\usepackage{dcolumn}        % lepší zarovnání sloupců v tabulkách
\usepackage{booktabs}       % lepší vodorovné linky v tabulkách
\usepackage{paralist}       % lepší enumerate a itemize
\usepackage[usenames]{xcolor}  % barevná sazba

\usepackage{url}

\usepackage{pdfpages}
%opening
\title{}
\author{}

\begin{document}
	\pagestyle{empty}
\section*{Abstrakt}
Webové e-mailové rozhrania sú v dnešnej dobe nenahraditeľnou súčasťou Internetu, hlavne vďaka ich jednoduchému používaniu pomocou webových prehliadačov a vďaka ľahkej integrácii veľkého množstva funkcií poskytnutých dodávateľom. Medzi najdôležitejšie funkcie, ktoré z časti alebo úplne chýbajú v open-source webových implementáciách, sú napríklad bezpriečinková organizácia pomocou tagov, navigácia poháňaná výkonným fulltextovým prehľadávaním a integrácia možností na spravovanie času. Táto práca popisuje novú, open-source alternatívu ku pokrokovým komerčným webovým e-mailovým rozhraniam, ktorá obsahuje tieto funkcie. Softvér integruje fulltextové vyhľadávanie poskytnuté ElasticSearch-ovou databázou do aktuálnej infraštruktúry na spracovanie e-mailov na UNIX-ových systémoch, na vytvorenie serverovej aplikácie, ktorá je používaná webovým užívateľským rozhraním, založeným na JavaScripte. Výkon výslednej aplikácie je testovaný na veľkom množstve e-mailov. 
\end{document}
