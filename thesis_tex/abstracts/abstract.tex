
%% Verze pro jednostranný tisk:
% Okraje: levý 40mm, pravý 25mm, horní a dolní 25mm
% (ale pozor, LaTeX si sám přidává 1in)
\documentclass[12pt,a4paper]{report}

% \openright zařídí, aby následující text začínal na pravé straně knihy
\let\openright=\clearpage

%% Pokud tiskneme oboustranně:
% \documentclass[12pt,a4paper,twoside,openright]{report}
% \setlength\textwidth{145mm}
% \setlength\textheight{247mm}
% \setlength\oddsidemargin{14.2mm}
% \setlength\evensidemargin{0mm}
% \setlength\topmargin{0mm}
% \setlength\headsep{0mm}
% \setlength\headheight{0mm}
% \let\openright=\cleardoublepage

%% Vytváříme PDF/A-2u
\usepackage[a-2u]{pdfx}

%% Přepneme na českou sazbu a fonty Latin Modern
\usepackage[slovak]{babel}
\usepackage{lmodern}
\usepackage[IL2]{fontenc}%T1
\usepackage{textcomp}
\usepackage{hyperref}

\usepackage{float}
\usepackage{subfigure}

%% Použité kódování znaků: obvykle latin2, cp1250 nebo utf8:
\usepackage[utf8]{inputenc}

%%% Další užitečné balíčky (jsou součástí běžných distribucí LaTeXu)
\usepackage{amsmath}        % rozšíření pro sazbu matematiky
\usepackage{amsfonts}       % matematické fonty
\usepackage{amsthm}         % sazba vět, definic apod.

%bolo treba vypnut kvoli rozdelovaniu
%\usepackage{bbding}         % balíček s nejrůznějšími symboly
% (čtverečky, hvězdičky, tužtičky, nůžtičky, ...)
\usepackage{bm}             % tučné symboly (příkaz \bm)
\usepackage{graphicx}       % vkládání obrázků
\usepackage{fancyvrb}       % vylepšené prostředí pro strojové písmo
\usepackage{indentfirst}    % zavede odsazení 1. odstavce kapitoly
%\usepackage{natbib}         % zajištuje možnost odkazovat na literaturu
% stylem AUTOR (ROK), resp. AUTOR [ČÍSLO]
\usepackage[nottoc]{tocbibind} % zajistí přidání seznamu literatury,
% obrázků a tabulek do obsahu
\usepackage{icomma}         % inteligetní čárka v matematickém módu
\usepackage{dcolumn}        % lepší zarovnání sloupců v tabulkách
\usepackage{booktabs}       % lepší vodorovné linky v tabulkách
\usepackage{paralist}       % lepší enumerate a itemize
\usepackage[usenames]{xcolor}  % barevná sazba

\usepackage{url}

\usepackage{pdfpages}
%opening
\title{}
\author{}

\begin{document}
	\pagestyle{empty}
\section*{Abstrakt}
Webmails are indisposable interfaces for working with the e-mail on the current Internet, mostly because of the simplicity of their deployment in browsers and easy integration with many provider-specific features. The most important features that are partially or fully missing in current open-source webmail implementations include directory-less mail organization by tags, navigation driven by a high-performance fulltext search, and integration of time-management capabilities. This thesis describes a new open-source alternative to advanced commercial webmails that possesses these features. The software integrates full-text search capabilities of the ElasticSearch database with current e-mail processing infrastructure on UNIX systems to create a back-end server application, which is used by a Javascript-based browser front-end. The performance of the solution is tested on a large e-mail dataset.
\end{document}
